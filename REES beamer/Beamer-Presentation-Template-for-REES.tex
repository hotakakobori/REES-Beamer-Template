% Options for packages loaded elsewhere
\PassOptionsToPackage{unicode}{hyperref}
\PassOptionsToPackage{hyphens}{url}
%
\documentclass[
  9pt,
  ignorenonframetext,
  compress]{beamer}
\usepackage{pgfpages}
\setbeamertemplate{caption}[numbered]
\setbeamertemplate{caption label separator}{: }
\setbeamercolor{caption name}{fg=normal text.fg}
\beamertemplatenavigationsymbolsempty
% Prevent slide breaks in the middle of a paragraph
\widowpenalties 1 10000
\raggedbottom
\setbeamertemplate{part page}{
  \centering
  \begin{beamercolorbox}[sep=16pt,center]{part title}
    \usebeamerfont{part title}\insertpart\par
  \end{beamercolorbox}
}
\setbeamertemplate{section page}{
  \centering
  \begin{beamercolorbox}[sep=12pt,center]{part title}
    \usebeamerfont{section title}\insertsection\par
  \end{beamercolorbox}
}
\setbeamertemplate{subsection page}{
  \centering
  \begin{beamercolorbox}[sep=8pt,center]{part title}
    \usebeamerfont{subsection title}\insertsubsection\par
  \end{beamercolorbox}
}
\AtBeginPart{
  \frame{\partpage}
}
\AtBeginSection{
  \ifbibliography
  \else
    \frame{\sectionpage}
  \fi
}
\AtBeginSubsection{
  \frame{\subsectionpage}
}
\usepackage{amsmath,amssymb}
\usepackage{lmodern}
\usepackage{ifxetex,ifluatex}
\ifnum 0\ifxetex 1\fi\ifluatex 1\fi=0 % if pdftex
  \usepackage[T1]{fontenc}
  \usepackage[utf8]{inputenc}
  \usepackage{textcomp} % provide euro and other symbols
\else % if luatex or xetex
  \usepackage{unicode-math}
  \defaultfontfeatures{Scale=MatchLowercase}
  \defaultfontfeatures[\rmfamily]{Ligatures=TeX,Scale=1}
\fi
\usefonttheme{serif}
% Use upquote if available, for straight quotes in verbatim environments
\IfFileExists{upquote.sty}{\usepackage{upquote}}{}
\IfFileExists{microtype.sty}{% use microtype if available
  \usepackage[]{microtype}
  \UseMicrotypeSet[protrusion]{basicmath} % disable protrusion for tt fonts
}{}
\makeatletter
\@ifundefined{KOMAClassName}{% if non-KOMA class
  \IfFileExists{parskip.sty}{%
    \usepackage{parskip}
  }{% else
    \setlength{\parindent}{0pt}
    \setlength{\parskip}{6pt plus 2pt minus 1pt}}
}{% if KOMA class
  \KOMAoptions{parskip=half}}
\makeatother
\usepackage{xcolor}
\IfFileExists{xurl.sty}{\usepackage{xurl}}{} % add URL line breaks if available
\IfFileExists{bookmark.sty}{\usepackage{bookmark}}{\usepackage{hyperref}}
\hypersetup{
  pdftitle={Your Presentation Title Here e.g.~Why REES is Best?},
  pdfauthor={Your Name},
  hidelinks,
  pdfcreator={LaTeX via pandoc}}
\urlstyle{same} % disable monospaced font for URLs
\newif\ifbibliography
\setlength{\emergencystretch}{3em} % prevent overfull lines
\providecommand{\tightlist}{%
  \setlength{\itemsep}{0pt}\setlength{\parskip}{0pt}}
\setcounter{secnumdepth}{-\maxdimen} % remove section numbering
\usetheme[]{Madrid}
 % Alternatively: miniframes, infolines, split
\useinnertheme{circles}

\definecolor{ualberta}{RGB}{0,124,65}
\setbeamercolor{structure}{fg=ualberta}

\setbeamertemplate{navigation symbols}{%
\insertslidenavigationsymbol
\insertframenavigationsymbol
\insertsubsectionnavigationsymbol
\insertsectionnavigationsymbol
\insertdocnavigationsymbol
\insertbackfindforwardnavigationsymbol
}
\mode<beamer>{\setbeamertemplate{blocks}[rounded][shadow=true]}
\setbeamercovered{transparent}
\setbeamercolor{block body example}{fg=blue, bg=black!20}

\useoutertheme{miniframes}


\institute[REES]{Supervised by \href{https://scholar.google.ca/citations?user=PmlfbK8AAAAJ&hl=en&oi=ao}{Professor Brent Swallow} and \href{https://scholar.google.ca/citations?user=6HdoCygAAAAJ&hl=en&oi=ao}{Professor Feng Qiu} \and \href{https://www.ualberta.ca/resource-economics-environmental-sociology/index.html}{Department of \textbf{R}esource \textbf{E}conomics and \textbf{E}nvironmental \textbf{S}ociology} \and ~ \and Our fullworking paper can be retrieved from here}

\addtobeamertemplate{title page}{}{\begin{center}\tiny{\emph{"The University of Alberta acknowledges that we are located on Treaty 6 territory, and respects the histories, languages, and cultures of First Nations, Métis, Inuit, and all First Peoples of Canada, whose presence continues to enrich our vibrant community."
}}\end{center}}

\titlegraphic{
    \includegraphics[width=4cm]{logo.jpg}}
    
\usepackage{amssymb}% http://ctan.org/pkg/amssymb
\usepackage{pifont}% http://ctan.org/pkg/pifont
\newcommand{\cmark}{\ding{51}}%
\newcommand{\xmark}{\ding{55}}%


\usepackage{tikz}
\usetikzlibrary{snakes}
\usepackage{rotating}
\usepackage{chronology}
\usepackage{xcolor}


\definecolor{links}{HTML}{0000FF}
\hypersetup{colorlinks,linkcolor=,urlcolor=links}
\AtBeginDocument{\title[Why REES ?]{Your Presentation Title Here e.g. Why REES is Best?}}
\ifluatex
  \usepackage{selnolig}  % disable illegal ligatures
\fi

\title{Your Presentation Title Here e.g.~Why REES is Best?}
\subtitle{Your Presentation Subtitle Here}
\author{Your Name}
\date{06 May, 2022}

\begin{document}
\frame{\titlepage}

\begin{frame}{Outline}
\protect\hypertarget{outline}{}
\label{agenda}

\textbf{Disclaimer/Announcement}\\

\begin{itemize}
\tightlist
\item
  Mention what needs to be announced beforehand\\

  \begin{itemize}
  \tightlist
  \item
    e.g.~all views/opinions presented are personal\\
  \end{itemize}
\end{itemize}

\textbf{Contents}

\begin{enumerate}
\item
  Executive Summary \hyperlink{1}{\beamergotobutton{ESummary}}\\
\item
  Motivation \hyperlink{2}{\beamergotobutton{Motivation}}\\
\item
  Method \hyperlink{3}{\beamergotobutton{Method}}\\
\item
  Result \hyperlink{4}{\beamergotobutton{Result}}\\
\item
  Conclusion \hyperlink{5}{\beamergotobutton{Conclusion}}\\
\item
  Reference \hyperlink{6}{\beamergotobutton{Reference}}\\
\end{enumerate}

\hfill\break

\begin{center}
\emph{Note} you can also jump to by clicking top bar 
\end{center}
\end{frame}

\hypertarget{executive-summary}{%
\section{Executive Summary}\label{executive-summary}}

\begin{frame}{Executive Summary}
\begin{itemize}
\item
  This is an easy-to-implement template REES students can use to
  present.
\item
  This is an introduction to Rmarkdown.
\end{itemize}
\end{frame}

\hypertarget{motivation-put-title-of-this-section-here}{%
\section{Motivation (Put title of this section
here)}\label{motivation-put-title-of-this-section-here}}

\begin{frame}{Motivation (Put title of this section here)}
\begin{itemize}
\item
  e.g.~REES can be more famous !
\item
  This is a list/item

  \begin{itemize}
  \tightlist
  \item
    This is a sub item
  \end{itemize}
\end{itemize}

\begin{enumerate}
\tightlist
\item
  This is a numbered list
\item
  You can also show one by one called \emph{increment}
\end{enumerate}

\begin{enumerate}[<+->]
\setcounter{enumi}{2}
\item
  See ?
\item
  This is increment
\item
  Note page number does not change?!
\end{enumerate}
\end{frame}

\begin{frame}
You can also \textbf{bolden} or \emph{italize} and such

See
\href{https://bookdown.org/yihui/rmarkdown/beamer-presentation.html}{this
page for more}
\end{frame}

\hypertarget{method}{%
\section{Method}\label{method}}

\begin{frame}{You can also draw}
\protect\hypertarget{you-can-also-draw}{}
\begin{center}
    \begin{tikzpicture}
        % draw horizontal line   
        \draw (-5,0) -- (6,0);
        % draw vertical lines
    \foreach \x in {-5,-4,-3,-2, -1,0,1,2,3,6}
    \draw (\x cm,2pt) -- (\x cm,-2pt);
     % draw nodes
    \draw (-5,0) node[below=3pt] {$ 21.12 $} node[above=3pt] {$\begin{turn}{45} Task A \end{turn}$};
    \draw (-4,0) node[below=3pt] {$ 22.1 $} node[above=3pt] {$\tiny \begin{turn}{75} Task B1, Task B2, Task B3 etc.. \end{turn}$};
    \draw (-3,0) node[below=3pt] {$ 2 $} node[above=3pt] {$\begin{turn}{45} Task C \end{turn}$};
    \draw (-2,0) node[below=3pt] {$ 3 $} node[above=3pt] {$\begin{turn}{45} Task D \end{turn}$};
    \draw (-1,0) node[below=3pt] {$ 4 $} node[above=3pt] {$\begin{turn}{45} Task E \end{turn}$};
    \draw (0,0) node[below=3pt] {$ 5 $} node[above=3pt] {$\begin{turn}{45} \end{turn}$};
    \draw (1,0) node[below=3pt] {$ 6 $} node[above=3pt] {$\begin{turn}{45} \end{turn}$};
    \draw (2,0) node[below=3pt] {$ 7 $} node[above=3pt] {$\tiny \begin{turn}{75} \end{turn}$};
    \draw (3,0) node[below=3pt] {$ 8 $} node[above=3pt] {$\begin{turn}{45} \end{turn}$};
    \draw (6,0) node[below=3pt] {$ 11 $} node[above=3pt] {$\begin{turn}{45} Goal \end{turn}$};
\end{tikzpicture}
\end{center}
\end{frame}

\begin{frame}{or make a table}
\protect\hypertarget{or-make-a-table}{}
\par

\begin{center}
\begin{tabular}{ c| c c}
\hline
    Method/Data & Same & Different \\ 
\hline
    Same & Reproducibility & Replicability\\ 
    Different & Robustness  & Generalizability\\ 
\hline
\end{tabular}
\end{center}
\end{frame}

\begin{frame}{or put an image}
\protect\hypertarget{or-put-an-image}{}
\begin{figure}

{\centering \includegraphics[width=0.75\linewidth,height=0.75\textheight]{rees-feature-850} 

}

\caption{caption goes here}\label{fig:unnamed-chunk-2}
\end{figure}
\end{frame}

\begin{frame}{or an equation}
\protect\hypertarget{or-an-equation}{}
\begin{block}{For example... Hansen (2.48)} 
$\begin{aligned} \beta &=\left(\mathbb{E}\left[X X^{\prime}\right]\right)^{-1} \mathbb{E}[X Y] \end{aligned}$
\end{block}
\end{frame}

\begin{frame}{}
\protect\hypertarget{section}{}
\begin{columns}
\column{0.5\textwidth}
- You can also do columns \par
- This section goes right \par
- See? 

\column{0.5\textwidth}
\begin{center} - This section goes left \par
\end{center}
\begin{figure}

{\centering \includegraphics[width=0.75\linewidth,height=0.5\textheight]{LO7Pg0M-_400x400} 

}

\caption{REES}\label{fig:unnamed-chunk-3}
\end{figure}

 \end{columns}
\end{frame}

\hypertarget{summary}{%
\section{Summary}\label{summary}}

\begin{frame}{Summary}
\begin{enumerate}[<+->]
\item
  REES
\item
  is
\item
  best !
\end{enumerate}
\end{frame}

\hypertarget{reference}{%
\section{Reference}\label{reference}}

\begin{frame}{Reference}
\begin{itemize}
\item
  \href{https://www.ssc.wisc.edu/~bhansen/econometrics/Econometrics.pdf}{\underline{Hansen's Econometrics}}
\item
  \href{https://www.moqixu.com/faq/faq-how-to-have-a-better-meeting-with-your-phd-supervisor}{\underline{Moqi's Blog}}
\end{itemize}
\end{frame}

\end{document}
